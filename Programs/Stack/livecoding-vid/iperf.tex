\documentclass{beamer}
\usepackage[utf8]{inputenc}
\usepackage{graphicx}
\usepackage{hyperref}

\title{iperf auf dem ITS BRD}
\author{Martin Becke}
\date{}

\begin{document}

\frame{\titlepage}

\begin{frame}
    \frametitle{Inhaltsverzeichnis}
    \tableofcontents
\end{frame}

\section{Wdh: lwIP-Projekt}
\begin{frame}
    \frametitle{Allgemein: lwIP-Projekt des RN ITS}
    \begin{itemize}
        \item lwIP (lightweight IP) ist ein Open-Source-Stack zur Implementierung des TCP/IP-Protokolls.
        \item Speziell für eingebettete Systeme mit begrenztem Speicher und Ressourcen entwickelt.
        \item Unterstützt TCP, UDP, ICMP, DHCP und PPP.
        \item Ermöglicht Netzwerkanbindung ohne hohe Speicheranforderungen.
    \end{itemize}
\end{frame}


\section{Ziel des iperf Projektes}
\begin{frame}
    \frametitle{Ziel des Projektes}
    \begin{itemize}
        \item Entwicklung einer Anwendung, die Basisdaten des Netzwerk-Interfaces erfasst.
        \item Nutzung von iPerf zur Durchsatzmessung und Analyse der Netzwerkempfindlichkeit.
        \item Kombination aus Delay-Messungen (Ping) und iPerf für detaillierte Performance-Analyse.
    \end{itemize}
\end{frame}

\section{Implementierungsübersicht}
\begin{frame}
    \frametitle{Implementierungsübersicht}
    \begin{itemize}
        \item Erstellung eines neuen Tasks namens \texttt{IPERF}.
        \item Integration der iPerf-Bibliothek von lwIP für TCP-basierte Kommunikation.
        \item Einbindung von \texttt{lwip/apps/lwiperf.h} zur Performance-Messung.
    \end{itemize}
\end{frame}

\section{Fortschritt und Ergebnisse}
\begin{frame}
    \frametitle{Fortschritt und Ergebnisse}
    \begin{itemize}
        \item Erstellung eines neuen Branches für iPerf-Integration.
        \item Verbesserung der Task-Management-Parameter für optimierte Kommunikation.
        \item Test: TCP-Server auf Port 5001 mit anfänglich > 1,5 Mbit/s. Warum?
    \end{itemize}
\end{frame}

\section{Optimierungsschritte}
\begin{frame}
    \frametitle{Optimierungsschritte}
    \begin{itemize}
        \item Exklusive Ausführung des iPerf-Tasks zur Maximierung der Leistung.
        \item Implementierung einer Callback-Funktion zur Rückmeldung auf dem Board.
        \item Ziel: Visualisierung und Bewertung der Systemleistungsgrenzen.
    \end{itemize}
\end{frame}

\section{Erweiterungen und GUI-Darstellung}
\begin{frame}
    \frametitle{Erweiterungen und GUI-Darstellung}
    \begin{itemize}
        \item Darstellung statischer und dynamischer Werte auf dem LCD-Bildschirm.
        \item Einsatz der \texttt{LCD GUI.h}-Bibliothek zur Steuerung der Benutzeroberfläche.
        \item Regelmäßige Aktualisierung dynamischer Werte für Echtzeit-Feedback.
    \end{itemize}
\end{frame}

\section{Nutzung von iPerf zur Netzwerkmessung}
\begin{frame}
    \frametitle{Nutzung von iPerf zur Netzwerkmessung}
    \begin{itemize}
        \item Nach Installation von iPerf: Test im Client-Modus mit \texttt{iperf -c 192.168.33.99 -t 300 -i 10}.
        \item \texttt{-c 192.168.33.99}: Verbindung mit dem Server.
        \item \texttt{-t 300}: Testdauer auf 300 Sekunden festgelegt.
        \item \texttt{-i 10}: Ausgabe der Zwischenstatistik alle 10 Sekunden.
    \end{itemize}
\end{frame}

\end{document} 
   